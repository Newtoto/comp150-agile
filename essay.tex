% Please do not change the document class
\documentclass{scrartcl}

% Please do not change these packages
\usepackage[hidelinks]{hyperref}
\usepackage[none]{hyphenat}
\usepackage{setspace}
\doublespace

% You may add additional packages here
\usepackage{amsmath}

% Please include a clear, concise, and descriptive title
\title{How can the daily stand-up be adjusted to accommodate for a project which lacks a central office space and common working hours?}

% Please do not change the subtitle
\subtitle{COMP150 - Agile Development Practice}

% Please put your student number in the author field
\author{1601016}

\begin{document}

\maketitle

\abstract{I will be finding solutions to the problems that arise from working in a university setting where students are working from different locations and work to different schedules. In order to find these solutions, I will be drawing parallels between this setting and that of global software development where workers operate mostly through online tools and have no real central workspace, this should allow me to draw new ideas into the classroom setting.}

\section{Introduction}

The daily stand up is a vital part of the scrum process and ensures that all team members are up to date, in a standard offiice setting this is something that can be easily scheduled and enforced, but, when workers are based in different locations with different working hours, issues arise, which means that the daily stand up needs to be adjusted to accommodate. In order to adjust the daily stand up for the scenario posed in the question, it is first important to understand the main and most common problems which are a result of working in different places at different times.

\section{Issues with remote working}

A increase in remote working [7] causes an increase in the amount of reliance on informal communication such as through email. [9][11] This direct contact from one team member to another results in a lack of knowledge sharing amongst the team of how the project is progressing [3]. The personal element also gets lost, meaning that the trust building that comes from regular communication [4] goes with it.
In order to avoid information being passed only between a couple of team members, a common communication platform needs to be implemented such as Slack so all the knowledge remains somewhere that the whole team can access. Face to face mobile communication should be encouraged in order to build trust between members when it isn’t possible to meet in person.


\section{Misaligned working hours}

Having misaligned working hours means that it is harder to schedule daily stand-ups since it will always result in some team members working outside their regular time in order to accommodate [6][12]. Also, because when a team member encounters a blocker which needs to be addressed by someone who works later in the day, they’ll have to wait for that other team member to start working in order to progress. This dead time reduces the overall working capacity of the team [3][5][7][10].
Daily stand ups are most effective when they include every member of the team since it aids communication and understanding, so organising these stand up meetings at a time where there is the most overlap in working hours is the best solution. If time zones are such that there is no overlap, there needs to be a way in which a worker can inform everyone of what they’re doing and if there are blockers. This can then be accessed by the rest of the team in working hours and addressed accordingly, for severe blockers there should be a way of communicating outside of working hours, but lining up another task to work on in the meantime will reduce the amount of dead time.

\section{Problems from the combination of both}

With the ever changing nature of agile development[31], it is vital that the team are kept up to date regularly with the requirements since knowledge transfer and sharing is impeded when there is a lack of communication [3]. If communication is unclear, which is more likely to happen remotely where tone and emotion is lost[28], this can lead to a lack of understanding of these requirements and wasted time on aspects that have been removed from the specification or developing a misunderstood aspect of the project.[4][2] This means that there must be clarity in explanations and regular updates need to be made in a place where the whole team has access to them. Calls are crucial when it comes to the daily stand ups, where tone and emotion can still carry through despite being remote. The communication is also faster than typing and it is more clear that people are present in the meeting. Using an online task board such as trello allows for team members to see all the available tasks and also what every else is working on, blockers can also be marked but it is still worth letting the scrum master know so it can be sorted out.

\section{Avoiding the magnification of existing issues}

Holding daily stand ups remotely also shares some of the issues which arise in a typical office environment. While some solutions can still be used, others need to be adjusted in order to apply to workers in different locations. Common issues are that workers arrive late, meetings draw on for longer than the recommended 15 minutes so workers lose focus, and dead time before and after the meeting [4].
A possible solution for late arrivals is just to not let them join [4], which links up to starting the meeting regardless of who’s present. This is something that would actually be easier online since access to a conference call can be revoked for latecomers. A consequence for the last to arrive can also be used to discourage lateness[4]. To enforce the 15 minutes a timer can be used[4], break down the 15 minutes to figure out how long each member gets and stick to it. Distractions are a bigger issue online since you can’t see what people are doing so keeping the meetings short is more crucial to hold interest. Dead time is actually less of an issue with remote working, team members can join the conference room a do work up to the point of the meeting and easily resume once it’s finished knowing everything will be in the same place.


\section{Conclusion}

Write your conclusion here. The conclusion should do more than summarise the essay, making clear the contribution of the work and highlighting key points, limitations, and outstanding questions. It should not introduce any new content or information.

\bibliographystyle{ieeetran}
\bibliography{references}

\end{document}
